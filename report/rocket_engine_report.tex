\documentclass{jsarticle}
\begin{document}
\begin{titlepage}
\title{ロケットエンジン設計書}
\date{2017/5/18}
\author{姫野 由宇、眞木 俊哉、三厨 航、蓑田 浩史}
\maketitle
\thispagestyle{empty}
\end{titlepage}




\section{ロケットの概要}
燃料は液体酸素・液体水素を用いた。$\rm{MR}=3.14$でペイロード比は最大値の$\Lambda = 0.0782$をとった。パラメータは以下のとおり。

\begin{tabbing} 
1. \=ペイロード比 ~~~~ \= $\Lambda = 0.07823$\\

2. \>燃焼室圧力 \> $P = 20 ~\rm{MPa}$\\

3. \>質量混合比 \>$\rm{MR}=3.14$\\

4. \>断熱火炎温度 \>$T_f = 2466 ~ \rm{K}$\\

5. \>比推力 \>$I_{\rm{sp}}= 429.41~\rm{s}$\\

6. \>燃焼室径\>$a = 1.0000 ~ \rm{m}$\\

\>燃焼室長さ\>$a = 0.1699 ~ \rm{m}$\\

7. \>タンク径\>$a = 4.0000 ~ \rm{m}$\\

\>タンク長さ\>$a = 6.7906 ~ \rm{m}$\\
\end{tabbing} 

\section{計算過程}
燃料は液体酸素・液体水素であるから、燃料$1~\rm{mol}$に対する燃焼後の燃料及び生成物のモル数をそれぞれ$n_{f,j}$、$n_p$として、
\[
M = \frac{n_{f,j}M_f+n_pM_p}{n_{f,j}+n_p}
\]
また、液体酸素・液体水素の密度はそれぞれ$1.14 ~ \rm{g/cm^3}$、$0.07 ~ \rm{g/cm^3}$であるから、
\[
f_{\rm{inert}} = \frac{1}{\big(\frac{1}{f_{\rm{inert,s}}}-1\big)\rho^*+1}
\]
で構造質量比が求まる。
一方、断熱火炎温度は、
\[
-n_p\Delta H_f \Big( 1-0.4 \frac{T_f-2500}{1000} \Big) = (n_{f,j}+n_p)\frac{\gamma}{\gamma-1} R_0 (T_f-300) ~~ (T_f > 2500)
\]
\[
-n_p\Delta H_f = (n_{f,j}+n_p) \frac{\gamma}{\gamma-1} R_0 (T_f-300) ~~ (T_f < 2500)
\]
で求められるから、出口速度$V_j$は
\[
V_j = \sqrt{2\eta_j C_p T_f \Big( 1-\Big( \frac{p_j}{p_0}\Big)^{\frac{\gamma}{\gamma-1}} \Big)}
\]
で表され、よって比推力$I_{\rm{sp}}$は
\[
I_{\rm{sp}} = \frac{V_j}{g}
\]
で求められる。
以上で求めた比推力と構造質量比から、ペイロード比が決定される。
\[
\Lambda = \frac{\exp \big( {\frac{-\Delta V}{gI_{\rm{sp} } }}\big)-f_{\rm{inert}}}{1-f_{\rm{inert}}}
\]
また、燃焼室長さ、タンク長さなどについては、...

\section{その他の燃料を用いた場合}
\subsection{液体酸素+液体フッ素}
ペイロード比は、質量混合比が$\rm{MR}=00$のとき、$\Lambda = 00$となった。
\begin{tabbing} 
1. \=ペイロード比 ~~~~ \= $\Lambda = 00$\\

2. \>燃焼室圧力 \> $P = 00 ~\rm{Pa}$\\

3. \>質量混合比 \>$\rm{MR}=111$\\

4. \>断熱火炎温度 \>$T_f = 000 ~ \rm{K}$\\

5. \>比推力 \>$I_{\rm{sp}}= 000 ~\rm{s}$\\

6. \>燃焼室径\>$a = 0 ~ \rm{m}$\\

\>燃焼室長さ\>$a = 0 ~ \rm{m}$\\

7. \>タンク径\>$a = 0 ~ \rm{m}$\\

\>タンク長さ\>$a = 0 ~ \rm{m}$\\
\end{tabbing} 


\subsection{ケロシン+酸素}
ペイロード比は、質量混合比が$\rm{MR}=00$のとき、$\Lambda = 00$となった。
\begin{tabbing} 
1. \=ペイロード比 ~~~~ \= $\Lambda = 00$\\

2. \>燃焼室圧力 \> $P = 00 ~\rm{Pa}$\\

3. \>質量混合比 \>$\rm{MR}=111$\\

4. \>断熱火炎温度 \>$T_f = 000 ~ \rm{K}$\\

5. \>比推力 \>$I_{\rm{sp}}= 000 ~\rm{s}$\\

6. \>燃焼室径\>$a = 0 ~ \rm{m}$\\

\>燃焼室長さ\>$a = 0 ~ \rm{m}$\\

7. \>タンク径\>$a = 0 ~ \rm{m}$\\

\>タンク長さ\>$a = 0 ~ \rm{m}$\\
\end{tabbing} 


\subsection{アンモニア+酸素}
ペイロード比は、質量混合比が$\rm{MR}=00$のとき、$\Lambda = 00$となった。
\begin{tabbing} 
1. \=ペイロード比 ~~~~ \= $\Lambda = 00$\\

2. \>燃焼室圧力 \> $P = 00 ~\rm{Pa}$\\

3. \>質量混合比 \>$\rm{MR}=111$\\

4. \>断熱火炎温度 \>$T_f = 000 ~ \rm{K}$\\

5. \>比推力 \>$I_{\rm{sp}}= 000 ~\rm{s}$\\

6. \>燃焼室径\>$a = 0 ~ \rm{m}$\\

\>燃焼室長さ\>$a = 0 ~ \rm{m}$\\

7. \>タンク径\>$a = 0 ~ \rm{m}$\\

\>タンク長さ\>$a = 0 ~ \rm{m}$\\
\end{tabbing} 


\subsection{ヒドラジン+酸素}
ペイロード比は、質量混合比が$\rm{MR}=00$のとき、$\Lambda = 00$となった。
\begin{tabbing} 
1. \=ペイロード比 ~~~~ \= $\Lambda = 00$\\

2. \>燃焼室圧力 \> $P = 00 ~\rm{Pa}$\\

3. \>質量混合比 \>$\rm{MR}=111$\\

4. \>断熱火炎温度 \>$T_f = 000 ~ \rm{K}$\\

5. \>比推力 \>$I_{\rm{sp}}= 000 ~\rm{s}$\\

6. \>燃焼室径\>$a = 0 ~ \rm{m}$\\

\>燃焼室長さ\>$a = 0 ~ \rm{m}$\\

7. \>タンク径\>$a = 0 ~ \rm{m}$\\

\>タンク長さ\>$a = 0 ~ \rm{m}$\\
\end{tabbing} 


\subsection{ヒドラジン+四酸化二窒素}
ペイロード比は、質量混合比が$\rm{MR}=00$のとき、$\Lambda = 00$となった。
\begin{tabbing} 
1. \=ペイロード比 ~~~~ \= $\Lambda = 00$\\

2. \>燃焼室圧力 \> $P = 00 ~\rm{Pa}$\\

3. \>質量混合比 \>$\rm{MR}=111$\\

4. \>断熱火炎温度 \>$T_f = 000 ~ \rm{K}$\\

5. \>比推力 \>$I_{\rm{sp}}= 000 ~\rm{s}$\\

6. \>燃焼室径\>$a = 0 ~ \rm{m}$\\

\>燃焼室長さ\>$a = 0 ~ \rm{m}$\\

7. \>タンク径\>$a = 0 ~ \rm{m}$\\

\>タンク長さ\>$a = 0 ~ \rm{m}$\\
\end{tabbing} 


\subsection{ヒドラジン+硝酸}
ペイロード比は、質量混合比が$\rm{MR}=00$のとき、$\Lambda = 00$となった。
\begin{tabbing} 
1. \=ペイロード比 ~~~~ \= $\Lambda = 00$\\

2. \>燃焼室圧力 \> $P = 00 ~\rm{Pa}$\\

3. \>質量混合比 \>$\rm{MR}=111$\\

4. \>断熱火炎温度 \>$T_f = 000 ~ \rm{K}$\\

5. \>比推力 \>$I_{\rm{sp}}= 000 ~\rm{s}$\\

6. \>燃焼室径\>$a = 0 ~ \rm{m}$\\

\>燃焼室長さ\>$a = 0 ~ \rm{m}$\\

7. \>タンク径\>$a = 0 ~ \rm{m}$\\

\>タンク長さ\>$a = 0 ~ \rm{m}$\\
\end{tabbing} 


\subsection{アルミニウム+過塩素酸アンモニウム}
ペイロード比は、質量混合比が$\rm{MR}=00$のとき、$\Lambda = 00$となった。
\begin{tabbing} 
1. \=ペイロード比 ~~~~ \= $\Lambda = 00$\\

2. \>燃焼室圧力 \> $P = 00 ~\rm{Pa}$\\

3. \>質量混合比 \>$\rm{MR}=111$\\

4. \>断熱火炎温度 \>$T_f = 000 ~ \rm{K}$\\

5. \>比推力 \>$I_{\rm{sp}}= 000 ~\rm{s}$\\

6. \>燃焼室径\>$a = 0 ~ \rm{m}$\\

\>燃焼室長さ\>$a = 0 ~ \rm{m}$\\

7. \>タンク径\>$a = 0 ~ \rm{m}$\\

\>タンク長さ\>$a = 0 ~ \rm{m}$\\
\end{tabbing} 



\end{document}